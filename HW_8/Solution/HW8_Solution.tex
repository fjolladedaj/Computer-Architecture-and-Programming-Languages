\documentclass[a4paper]{article}
\usepackage[pdftex]{hyperref}
\usepackage[latin1]{inputenc}
\usepackage[english]{babel}
\usepackage{a4wide}
\usepackage{amsmath}
\usepackage{amssymb}
\usepackage{algorithmic}
\usepackage{algorithm}
\usepackage{ifthen}
\usepackage{listings}
\usepackage{array}
\usepackage{tabu}
% move the asterisk at the right position
\lstset{basicstyle=\ttfamily,tabsize=4,literate={*}{${}^*{}$}1}
%\lstset{language=C,basicstyle=\ttfamily}
\usepackage{moreverb}
\usepackage{palatino}
\usepackage{multicol}
\usepackage{tabularx}
\usepackage{comment}
\usepackage{verbatim}
\usepackage{color}
\usepackage{graphicx}
\usepackage{array,mathtools}
\usepackage{amsmath}
\usepackage{enumitem,amssymb}
\usepackage{pifont}
\newlist{todolist}{itemize}{2}
\setlist[todolist]{label=$\square$}
\newcommand{\cmark}{\ding{51}}%
\newcommand{\xmark}{\ding{55}}%
\newcommand{\done}{\rlap{$\square$}{\raisebox{2pt}{\large\hspace{1pt}\cmark}}%
\hspace{-2.5pt}}
\newcommand{\wontfix}{\rlap{$\square$}{\large\hspace{1pt}\xmark}}



%% pdflatex?
\newif\ifpdf
\ifx\pdfoutput\undefined
\pdffalse % we are not running PDFLaTeX
\else
\pdfoutput=1 % we are running PDFLaTeX
\pdftrue
\fi
\ifpdf
\fi
\ifpdf
\DeclareGraphicsExtensions{.pdf, .jpg}
\else
\DeclareGraphicsExtensions{.eps, .jpg}
\fi

\parindent=0cm
\parskip=0cm

\setlength{\columnseprule}{0.4pt}
\addtolength{\columnsep}{2pt}

\addtolength{\textheight}{5.5cm}
\addtolength{\topmargin}{-26mm}
\pagestyle{empty}

%%
%% Sheet setup
%% 
\newcommand{\coursename}{Computer Architecture and Programming Languages}
\newcommand{\courseno}{CO20-320241}
\newcommand*{\carry}[1][1]{\overset{#1}}
\newcolumntype{B}[1]{r*{#1}{@{\,}r}}
 
\newcommand{\sheettitle}{Homework}
\newcommand{\mytitle}{}
\newcommand{\mytoday}{{11th of November}, 2019}

% Current Assignment number
\newcounter{assignmentno}
\setcounter{assignmentno}{8}

% Current Problem number, should always start at 1
\newcounter{problemno}
\setcounter{problemno}{1}

%%
%% problem and bonus environment
%%
\newcounter{probcalc}
\newcommand{\problem}[2]{
  \pagebreak[2]
  \setcounter{probcalc}{#2}
  ~\\
  {\large \textbf{Problem \textcolor{blue}{\arabic{assignmentno}}.\textcolor{blue}{\arabic{problemno}}} \hspace{0.2cm}\textit{#1}} \refstepcounter{problemno}\vspace{2pt}\\}

\newcommand{\bonus}[2]{
  \pagebreak[2]
  \setcounter{probcalc}{#2}
  ~\\
  {\large \textbf{Bonus Problem \textcolor{blue}{\arabic{assignmentno}}.\textcolor{blue}{\arabic{problemno}}} \hspace{0.2cm}\textit{#1}} \refstepcounter{problemno}\vspace{2pt}\\}

%% some counters  
\newcommand{\assignment}{\arabic{assignmentno}}

%% solution  
\newcommand{\solution}{\pagebreak[2]{\bf Solution:}\\}

%% Hyperref Setup
\hypersetup{pdftitle={Homework \assignment},
  pdfsubject={\coursename},
  pdfauthor={},
  pdfcreator={},
  pdfkeywords={Computer Architecture and Programming Languages},
  %  pdfpagemode={FullScreen},
  %colorlinks=true,
  %bookmarks=true,
  %hyperindex=true,
  bookmarksopen=false,
  bookmarksnumbered=true,
  breaklinks=true,
  %urlcolor=darkblue
  urlbordercolor={0 0 0.7}
}

\begin{document}


\coursename \hfill Course: \courseno\\
Jacobs University Bremen \hfill \mytoday\\
Fjolla Dedaj\hfill
\vspace*{0.3cm}\\
\begin{center}
{\Large \sheettitle{} \textcolor{blue}{\assignment}\\}
\end{center}
\problem{}{0}
Compute the IEEE 754 single precision binary representation of the following numbers:\\
\\
(i) $\frac{25}{32}$\\
\\
(ii) $27.3515625$\\
\\
\solution
\\
(i) $\frac{25}{32} = \frac{25}{2^{5}} = \frac{11001}{2^5} = 0.11001 = 1.1001 \times 2^{-1}$\\
\\
Exponent = 127 + (-1) = 126 = 1111110\\
Sign = 0 (positive number)\\
Fraction = 10010000000000000000000\\
\\
\begin{tabular}{|c|c|c|}
        \hline
        $S$  &   $E$(exponent) &   $F$(fraction) \\ \hline
        $0$   &   $01111110$ &   $10010000000000000000000$ \\ \hline
    \end{tabular}
\\
\\
\\
(ii) $27.3515625$\\
\\
$27 = 11011_{2}$\\
\\
\begin{tabular}{|c|c|c|}
        \hline
        $0.3515625 \times 2$  &   $0.703125$ &   0 \\ \hline
        $0.703125 \times 2$   &   $1.40625$ &   1 \\ \hline
        $0.40625 \times 2$   &   $0.8125$ &   0  \\ \hline
        $0.8125 \times 2$   &   $1.625$ &   1 \\ \hline
        $0.625 \times 2$   &   $1.25$ & 1 \\ \hline
        $0.25 \times 2$ &   $0.5$ &  0 \\ \hline
        $0.5 \times 2$   & 1 & 1 \\ \hline
    \end{tabular}
\\
\\
\\
$27.3515625 = 11011.01011010...0 = 1.1011010110100...0 \times 2^{4}$\\
$E = 127 + 4 = 131_{10} = 10000011_2$\\
\\
\begin{tabular}{|c|c|c|}
        \hline
        $S$  &   $E$(exponent) &   $F$(fraction) \\ \hline
        $0$   &   $10000011$ &   $10110101101000000000000$ \\ \hline
    \end{tabular}
\\
\\
\problem{}{0}
\solution
\begin{itemize}
  \item[\done] = True
  \item[\wontfix] = False
  \item MIPS properties:
  \begin{todolist}
  \item[\done] MIPS has an alignment restriction, that means words must start at addresses
that are multiples of 4.
  \item[\wontfix] All MIPS instructions are 30 bits long.
  \item[\wontfix] MIPS can perform arithmetic operations on memory locations.
  \item[\wontfix] Parameters to functions are always passed via the stack.
  \item[\wontfix] A procedure jumps to the address stored in the stack pointer register after it finishes execution.
  \end{todolist}
\end{itemize}

\pagebreak

\problem{}{0}
In MIPS assembly language, registers \$s0 to \$s7 map onto register 16 to 23, and registers \$t0 to \$t7 map onto register 8 to 15. The opcode for addition and subtraction is 0. The function code is 32 for addition and 34 for subtraction. Given all this, translate the following binary word into a MIPS instruction:\\
\\
$000000$ $10000$ $10101$ $01011$ $00000$ $100000$\\
\\
\solution
\\
\begin{center}
\begin{tabular}{|c|c|c|c|c|c|}
        \hline
        Op code  & Source 1 &   Source 2 & Destination & Shift Amount & Fraction Code \\ \hline
        $000000$   &   $10000$  &   $10101$ & $01011$ & $00000$ & $100000$  \\ \hline
    \end{tabular}
\\
\end{center}
\begin{center}
$\Downarrow$
\end{center}
\begin{center}
\begin{tabular}{|c|c|c|c|c|c|}
        \hline
        0  & 16 & 21 & 11 & 0 & 32 \\ \hline
        add/subtract  &   \$s0 &   \$s5 & \$t3 & No shift & add \\ \hline
    \end{tabular}
\end{center}
Thus, the instruction is:\\
\begin{center}
\textbf{add \$t3, \$s0, \$s5}\\
\end{center}
\problem{}{0}
\\
\textbf{a) How many bits can be used for the destination address in the j (jump) instruction?\\}
\\
\solution
There is room for a 26-bit address. The 26-bit target address field is transformed into a 32-bit address. This is done at run-time, as the jump instruction is executed.\\
\\
\textbf{b) If the address representation within the instruction does not cover the full 32-bit range (as in the previous question), what can be done to still be able to jump anywhere (assuming 32-bit addresses)}?\\
\\
\solution
As mentioned earlier, the 26-bit target address field can be transformed into a 32-bit address. This is done at run-time, as the jump instruction is executed. jr instruction has this behavior.
\\
\problem{}{0}
\solution
\\
\begin{center}
\begin{tabular}{|c|c|c|c|c|c|}
        \hline
        Class  & CPI on P1 & Freq * CPI on P1 & CPI on P2 & Frequency & Freq * CPI on P2\\ \hline
        A   & 1 & 0.6 & 2 & 60\% & 1.2\\ \hline
        B   & 2 & 0.2 & 2 & 10\% & 0.2\\ \hline
        C   & 3 & 0.3 & 2 & 10\% & 0.2\\ \hline
        D   & 4 & 0.4 & 4 & 10\% & 0.4\\ \hline
        E   & 3 & 0.3 & 4 & 10\% & 0.4\\ \hline
           &  & $\sum = 1.8$ &  &  & $\sum = 2.4$\\ \hline
    \end{tabular}
\\
\end{center}
CPU time for P1= $\frac{\textrm{Instruction Count x CPI}}{\textrm{Clock Rate}}$ = $\frac{13 \times 1.8 }{4} = 5.85$\\
\\
CPU time for P2= $\frac{\textrm{Instruction Count x CPI}}{\textrm{Clock Rate}}$ = $\frac{14 \times 2.4 }{6} = 5.6$\\
\\
Therefore, 5.85/5.6 $\approx$ 1.045 meaning that P2 is 4.5\% faster and will render the image first.\\

\problem{}{0}
\solution
\\
\begin{center}
\begin{tabular}{|c|c|c|}
        \hline
        Class  & CPI on P1 &  CPI on P2 \\ \hline
        A   & 1 & 2 \\ \hline
        B   & 3 & 3 \\ \hline
        C   & 3 & 2 \\ \hline
        D   & 4 & 3 \\ \hline
        E   & 2 & 3 \\ \hline
    \end{tabular}
\\
\end{center}


T(P1)/T(P2) =(1*2+3+3+4+2)4/(2*2+3+2+3+3)2 = 14*4/15*2 = 28/15 $\approx$ 1.866\\
\\
Instruction count is not included because it is the same, therefore the ratio doesn't change.\\
\\
Therefore, P2 is 86.6\% faster.\\

\end{document}